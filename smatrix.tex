\documentclass[12pt]{article}
\usepackage{braket}
\usepackage{physics}
\usepackage{graphicx}
\usepackage{times}
\usepackage[export]{adjustbox}
\usepackage{listings}
\usepackage{mathcomp}
\usepackage{hyperref}
\usepackage{bm,amsmath}
\usepackage{amssymb}
\usepackage{float}
\usepackage{indentfirst}
\usepackage{bigints}
\usepackage{listings}
\usepackage{color}
\hypersetup{
colorlinks=true,
linkcolor=blue,
filecolor=magenta,
urlcolor=cyan,
pdftitle={Overleaf Example},
pdfpagemode=FullScreen,
}
\definecolor{dkgreen}{rgb}{0,0.6,0}
\definecolor{gray}{rgb}{0.5,0.5,0.5}
\definecolor{mauve}{rgb}{0.58,0,0.82}
\lstset{frame=tb,
language=Python,
aboveskip=3mm,
belowskip=3mm,
stepnumber = 1,
showstringspaces=false,
columns=flexible,
basicstyle={\small\ttfamily},
numbers=left,
numberstyle=\color{gray},
keywordstyle=\color{blue},
commentstyle=\color{dkgreen},
stringstyle=\color{mauve},
breaklines=true,
breakatwhitespace=true,
tabsize=3
}
\numberwithin{equation}{section}

\title{S-matrix}
\author{Ting-Kai Hsu}
\date{\today}

\begin{document}
\maketitle
\tableofcontents
\section{Interacting Field Theory}
Let's try to figure out the \textit{two-point function} or \textit{two-point correlation Green's function},
\begin{equation}
    \langle\Omega|T\phi(x)\phi(y)|\Omega\rangle
\end{equation}\label{1.1}
Note that $\Omega$ is ground state and $x$ and $y$ are spacetime four vectors with $T$ denoting the time-ordering operator\footnote{I couldn't figure out why we need this operator here.}.
One must be careful that the ground state $\ket*{\Omega}$ here is different with the ground state $\ket*{0}$ in free theory\footnote{Where I don't know why?}.
The correlation function can be seen as the amplitude of the propagation for a particle between $y$ and $x$\footnote{Does this mean there is only spacetime distance difference in these states?}.
\subsection{Free Theory}
The lagrangian,
\begin{equation}
    \mathcal{L} = (\partial_{\mu}\phi)^2 - m^{2}\phi^2
\end{equation}
The two-point function of free field could be directly computed by the solution of the field,
\begin{equation}
    \phi(\mathbf{x}) = \int{\frac{d^3p}{(2\pi)^3}\frac{1}{\sqrt{2E_{\mathbf{p}}}}\left(a_{\mathbf{p}}e^{-i\mathbf{p}\cdot\mathbf{x}} + a^{\dagger}_{\mathbf{p}}e^{i\mathbf{p}\cdot\mathbf{x}}\right)}
\end{equation}\label{1.3}
This is the form in Schr$\ddot{\text{o}}$dinger picture, and next we should consider the time evolution of this field operator, that is, in Heisenberg's picture.
\begin{equation}
    \phi(x) = e^{iHt}\phi(\mathbf{x})e^{-iHt}
\end{equation}
Also note that the creator and annihilator would have the following relation with the hamiltonian of the system,
\begin{equation}
    \begin{split}
        Ha_{\mathbf{p}} = a_{\mathbf{p}}\left(H-E_{\mathbf{p}}\right)\\
        Ha_{\mathbf{p}}^{\dagger} = a_{\mathbf{p}}^{\dagger}\left(H+E_{\mathbf{p}}\right)
    \end{split}
\end{equation}
These are the properties of the annihilator and creator, which are related to SHO in quantum mechanics.
Thus we have,
\begin{equation}
    \begin{split}
        e^{iHt}a_{\mathbf{p}}e^{-iHt} = a_{\mathbf{p}}e^{i(H-E_{\mathbf{p}})t}e^{-iHt} = a_{\mathbf{p}}e^{-iE_{\mathbf{p}}t}\\
        e^{iHt}a^{\dagger}_{\mathbf{p}}e^{-iHt} = a_{\mathbf{p}}e^{i(H+E_{\mathbf{p}})t}e^{-iHt} = a_{\mathbf{p}}e^{iE_{\mathbf{p}}t}
    \end{split}
\end{equation}
Then eq(\hyperref[1.3]{1.3}) becomes,
\begin{equation}
    \phi(x) = \int{\frac{d^3p}{(2\pi)^3}\frac{1}{\sqrt{2E_{\mathbf{p}}}}\left(a_{\mathbf{p}}e^{ip\cdot x} + a^{\dagger}_{\mathbf{p}}e^{ip\cdot x}\right)}
\end{equation}
with $p_0 = E_{\mathbf{p}}$
\\\indent \textit{\textbf{Remark:}} On one hand, the field is written as a Hilbert space operator $a_{\mathbf{p}} \text{ and }a^{\dagger}_{\mathbf{p}}$, which creates and destroys the particles that are quanta of field excitation.
On the other hand, field is written as a linear combination of solutions $e^{ip\cdot x}$ and $e^{-ip\cdot x}$ of the Klein-Gordon equation. 
Note that a positive-frequency solution of the field with its coefficient, the operator that \textit{destroys} a particle in the single-particle waveform. If we view $\phi(x)$ as wave function, and then we see the part with negative frequency carrying creator of the field and the positive frequency carrying annihilator of the field.
Please note that here \textit{negative} and \textit{positive} doesn't mean the energy or frequency is really negative, in fact, they should always be positive\footnote{I don't know how this point of view can be created, and why it solves the paradox of KG equation.}.
\subsection{Causality}

\section{Perturbation Expansion}
Back to eq(\hyperref[1.1]{1.1}).
As we did the free field theory, we consider the field in Heisenberg's picture,
\begin{equation}
    \phi(x) = e^{iH(t-t_0)}\phi(t_0,\mathbf{x})e^{-iH(t-t_0)}
\end{equation}
where the hamiltonian is the sum of free hamiltonian part and interaction hamiltonian part,
\begin{equation}
    H = H_0 + H_{\text{int}} = H_0 + \int\,d^3x\,\frac{\lambda}{4!}\phi^{4}(t_0,\mathbf{x})
\end{equation}
If we consider a special case that no interaction exist ($\lambda = 0$), we got,
\begin{equation}
    \phi_{\text{I}}(x) = e^{iH_0(t-t_0)}\phi(t_0, \mathbf{x})e^{-iH_0(t-t_0)} = \int{\frac{d^3p}{(2\pi)^3}\frac{1}{\sqrt{2E_{\mathbf{p}}}}\left(a_{\mathbf{p}}e^{-ip\cdot x} + a^{\dagger}_{\mathbf{p}}e^{ip\cdot x}\right)}
\end{equation}\label{2.3}
Note that in the integral we have $x^0 = t - t_0$ and call this quantity \textit{interaction picture field}.
Now we try to express the original field in Heisenberg's picture in terms of interaction picture field.
\begin{equation}
    \phi(x) = e^{iH(t-t_0)}e^{-iH_0(t-t_0)}\phi_{\text{I}}(x)e^{iH_0(t-t_0)}e^{-iH(t-t_0)}
\end{equation}
If we rewrite it and define a unitary operator,
\begin{equation}
    \begin{split}
        U(t, t_0) = e^{iH_0(t-t_0)}e^{-iH(t-t_0)}\\
        \phi_{\text{I}}(x) = U(t,t_0)\phi(x)U^{\dagger}(t,t_0)
    \end{split}
\end{equation}
known as the interaction picture propagator or time-evolution operator. 
In quantum mechanics, we know that time-evolution operator would statisfies Schr$\ddot{\text{o}}$dinger's equation.
\begin{equation}
    \begin{split}
        i\frac{\partial}{\partial t}U(t, t_0) = i\left(iH_0e^{iH_0(t-t_0)}e^{-iH(t-t_0)}-ie^{iH_0(t-t_0)}He^{-iH(t-t_0)}\right)\\
        =e^{iH_0(t-t_0)}(H-H_0)e^{-iH(t-t_0)}\\
        =e^{iH_0(t-t_0)}(H_{\text{int}})\,e^{-iH_0(t-t_0)}\,e^{iH_0(t-t_0)}e^{-iH(t-t_0)}\\
        =H_{\text{int, I}}(t)U(t, t_0)
    \end{split}
\end{equation}\label{2.6}
The definition of interaction hamiltonian in interaction picture is similar to eq(\hyperref[2.3]{2.3}),
\begin{equation}
    H_{\text{int. I}}(t) = e^{iH_0(t-t_0)}H_{\text{int, I}}e^{-iH_0(t-t_0)} = \int{d^3x\frac{\lambda}{4!}\phi_{\text{I}}^{4}(x)}
\end{equation}
Next, we should be able to find the explicit solution to eq(\hyperref[2.6]{2.6}) with initial condition $U(t_0, t_0) = 1$, by iterating the differential equation.
\begin{equation}
    U(t,t_0) = 1-i\int_{t_0}^{t}{dt_1\,H_{\text{int, I}}(t_1)U(t_1, t_0)}
\end{equation}
\[ = 1-i\int_{t_0}^{t}{dt_1\,H_{\text{int, I}}(t_1)\left(1-i\int_{t_0}^{t_1}{dt_2\,H_{\text{int, I}}(t_2)U(t_2, t_0)}\right)} \]
\[ = 1+(-i)\int_{t_0}^{t}{dt_1\,H_{\text{int, I}}(t_1)} + (-i)^2\int_{t_0}^{t}dt_1\int_{t_0}^{t_1}dt_2\,H_{\text{int, I}}(t_1)H_{\text{int, I}}(t_2)+\cdots\]
Now consider \textit{time order operator},
\begin{equation}
    \int_{t_0}^{t}dt_1\int_{t_0}^{t_1}dt_2{\,H_{\text{int, I}}(t_1)H_{\text{int, I}}(t_2)} = \frac{1}{2}\int_{t_0}^{t}dt_1\int_{t_0}^{t}dt_2{T\left\{H_{\text{int, I}}(t_1)H_{{\text{int, I}}}(t_2)\right\}}
\end{equation}
Generally, we get,


\section{In Out State}
To study S-matrix, that is, the scattering process in quantum field theory, one must know what is "In" and "Out" state.
These physical quantities are related to the interacting field and process, which is of importance.
In interacting field theory, we have to use nonlinear term of hamiltonian and lagrangian, with different Fourier modes that represent the different particle which can occupy them respectively.
In order to \textbf{preserve causality}, that is, to make sure the formalism obey the principle of relativity, we must have the products of fields at the \textbf{same spacetime point}.
We've already learnt the scattering theory in quantum mechanics, and we would review them in this section.
\\\indent First we separate the total hamiltonian into free hamiltonian (or one may say asymptotic hamiltonian) and interacting hamiltonian.
\begin{equation}
    H(t) = H_{0}(t) + H_{\text{int}}(t)
\end{equation}
Suppose we're in Heisenberg's picture.

\end{document}